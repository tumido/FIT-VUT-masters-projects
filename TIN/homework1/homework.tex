\documentclass[11pt, a4paper]{article}
\usepackage[czech]{babel}
\usepackage[utf8x]{inputenc}
\usepackage[T1]{fontenc}
\usepackage{times}
\usepackage[left=1.5cm, top=2.5cm, text={18cm, 25cm}]{geometry}
\usepackage{enumitem}
\usepackage{amsmath, amssymb, amsthm, amsfonts}
\usepackage{multicol}
\usepackage{graphicx}
\usepackage{auto-pst-pdf}
\usepackage{multirow}


\title{TIN\,--\,Domácí úkol 1}
\author{Tomáš Coufal, xcoufa09@stud.fit.vutbr.cz}
\newcommand{\myuv}[1]{\quotedblbase#1\textquotedblleft}

\begin{document}
\maketitle
\section{Příklad}
\subsection*{Zadání}
Pro jazyk $L_1 = \{a^ib^jc^k\ |\ (j \ne k) \vee (j = 2l, l \geq 0)\}$ sestavte gramatiku $G_1$ takovou, že $L(G_1) = L_1$.
Určete typ $G_1$ a $L_1$ dle Chomského hierarchie jazyků a rozhodněte, zda-li se tyto typy mohou obecně lišit.
\subsection*{Řešení}
Pro jazyk $L_1$ sestavíme gramatiku $ G_1 = \{N, T, P, S\} $, kde:
\begin{enumerate}
	\item $ N = \{S, A, B, C, D, E, F\} $
	\item $ T = \{a, b, c\} $
	\item $ P $:
	      \begin{itemize}[label={},noitemsep]
	      	\item $ S \rightarrow ABC\ |\ aD\ |\ Ec $
	      	\item $ A \rightarrow aA\ |\ \varepsilon $
	      	\item $ B \rightarrow bbB\ |\ \varepsilon $
	      	\item $ C \rightarrow cC\ |\ \varepsilon$
	      	\item $ D \rightarrow aDc\ |\ aD\ |\ F$
	      	\item $ E \rightarrow aEc\ |\ Ec\ |\ F$
	      	\item $ F \rightarrow bF\ |\ \varepsilon$
	      \end{itemize}
	      Množina pravidel může být v různých řešeních různá, zde jsem postupoval rozdělením podle předpisu přijímaného
	      jazyka, který předpokládá 3 možné situace:
	      \begin{itemize}[label={-},noitemsep]
	      	\item Sudý počet terminálů $b$ (popsán pomocí neterminálů $A, B, C$)
	      	\item Počet terminálů $a$ je větší než počet terminálů $c$ (popsán pomocí neterminálů $D, F$)
	      	\item Počet terminálů $c$ je větší než počet terminálů $f$ (popsán pomocí neterminálů $E, F$)
	      \end{itemize}
\end{enumerate}

Z důvodů podmínky v definici jazyka $L_1$, kdy $i \neq k$, tj. že počet terminálů $a$ a $c$ si není roven, je
potřeba pravidel jako například $ D \rightarrow aDc $. To znamená, že námi popisovaná gramatika vyhovuje v Chomského
klasifikaci typu 2. Pro typ 3 nejsou taková pravidla povolena. Námi definovaná gramatika vede na implementaci
zásobnikovým automatem. Z těchto příčin je jak gramatika $G_1$, tak i jazyk $L_1$ typu 2 podle Chomského klasifikace
jazyků.

Tvrzení, o možnosti, aby se třída jazyka lišila oproti gramatice je pravdivé a vyplývá z vět 2.3 a 2.4 ve skriptech.
Z těch plyne, že každou gramatiku $n$-tého typu lze převést na gramatiku typu $n - 1$. To znamená, že pro ten samý jazyk
mohou existovat gramatiky různých typů.

\section{Příklad}
\subsection*{Zadání}
Regulární výraz: $r_2 = (a + b + \varepsilon)^*(ab)^*$. Převeďte postupně (přes RKA, DKA) na redukovaná DKA $M_2$. Pro
jazyk $L(M_2)$ určete počet tříd ekvivalence relace $\sim_L$, vypište je a popište např. pomocí KA.
\subsection*{Řešení}
Nejprve pro zadaný RV vytvoříme strom:
\begin{center}
	\includegraphics[width=4in]{exercise_2_tree.ps}
\end{center}

\noindent Podle tohoto stromu, jsme schopni algoritmicky sestavit rozšířený konečný automat:
\begin{center}
	\includegraphics[width=6.8in]{exercise_2_rka.ps}
\end{center}


\noindent $ M_{RKA} = \{Q_{RKA}, \Sigma, \delta_{RKA}, S_{RKA}, F_{RKA}\}$, kde:
\begin{enumerate}
	\item $ Q_{RKA} = \{1, 2, 3, 4, 5, 6, 7, 8, 9, 10, 11, 12, 13, 14, 15\}$
	\item $ \Sigma = \{a, b\}$
	\item $ \delta_{RKA} $ viz nákres automatu
	\item $ S_{RKA} = 1$
	\item $ F_{RKA} = \{15\}$
\end{enumerate}

\noindent Pro jeho determinizaci si musíme spočítat $\varepsilon$\textit{-uzávěr} pro každý stav a přijímaný znak:
\begin{itemize}[label={},noitemsep]
	\item $\varepsilon$\textit{-uzávěr}$(1) = \{1, 2, 3, 4, 11, 5, 6, 8, 12, 13, 16\} = A$
	\item $\delta'(A, a) = \varepsilon$\textit{-uzávěr}$(\{7, 14\}) = \{7, 10, 11, 2, 3, 4, 5, 6, 8, 12, 13, 14, 16\} = B$
	\item $\delta'(A, b) = \varepsilon$\textit{-uzávěr}$(\{9\}) = \{9, 10, 11, 2, 3, 4, 5, 6, 8, 12, 13, 16\} = C$
	\item $\delta'(B, a) = \varepsilon$\textit{-uzávěr}$(\{7, 14\}) = B$
	\item $\delta'(B, b) = \varepsilon$\textit{-uzávěr}$(\{9, 15\}) = \{9, 10, 11, 2, 3, 4, 5, 6, 8, 12, 13, 15, 16\} = D$
	\item $\delta'(C, a) = \varepsilon$\textit{-uzávěr}$(\{7, 14\}) = B$
	\item $\delta'(C, b) = \varepsilon$\textit{-uzávěr}$(\{9\}) = C$
	\item $\delta'(D, a) = \varepsilon$\textit{-uzávěr}$(\{7, 14\}) = B$
	\item $\delta'(D, b) = \varepsilon$\textit{-uzávěr}$(\{9\}) = C$
\end{itemize}

\noindent Z vypočtených uzávěrů plyne, že po determinizaci nám vznikne automat $M_{DKA} = \{Q_{DKA}, \Sigma, \delta_{DKA}, S_{DKA}, F_{DKA}\}$ takový, že:
\begin{enumerate}
	\item $ Q_{DKA} = \{A, B, C, D\}$
	\item $ \Sigma = \{a, b\}$
	\item $ \delta_{DKA} $ viz nákres automatu
	\item $ S_{DKA} = A$
	\item $ F_{DKA} = \{a, B, C, D\}$
\end{enumerate}
\begin{center}
	\includegraphics[width=3in]{exercise_2_dka.ps}
\end{center}

\noindent Následně jej můžeme redukovat pomocí výpočtu tříd ekvivalence:

\begin{multicols}{2}
	$\equiv^0 \{\{A, B, C, D\}\}$

	$\equiv^1 \{\{A, B, C, D\}\} = \equiv^0 = \equiv$

	$Q' = \{[A]\}$, kde $[A] = \{A, B, C, D\}$
	\columnbreak

	\begin{tabular}{|cc|cc|}
		\hline
		\multicolumn{2}{|c}{$\equiv^0$}& $a$ & $b$ \\
		\hline
		\multirow{4}{*}{0:} & $A$ & $B_0$ & $C_0$ \\
		                    & $B$ & $B_0$ & $D_0$ \\
		                    & $C$ & $B_0$ & $C_0$ \\
		                    & $D$ & $B_0$ & $C_0$ \\
		\hline
	\end{tabular}
\end{multicols}

\noindent Výsledný redukovaný deterministický konečný automat tedy vypadá takto:
\begin{center}
	\includegraphics[width=0.8in]{exercise_2_rdka.ps}
\end{center}

\noindent Z výpočtu výše nám vyplývá počet tříd ekvivalence, tak RV pro přijímaný jazyk, automat viz výše:

$L^{-1}([A]) = \varepsilon + (a + b)^*$

\section{Příklad}
\subsection*{Zadání}
NKA $M_3$ nad abecedou $ \Sigma = \{a, b, c\} $:
\begin{center}
	\includegraphics[width=2.5in]{exercise_3.ps}
\end{center}

\subsection*{Řešení}
Sestavíme rovnice pro jednotlivé stavy:
\begin{itemize}[label={},noitemsep]
	\item $ A = aB + cB $
	\item $ B = cB + bC $
	\item $ C = aB + bA + \varepsilon $
\end{itemize}
Poté spočítáme RV pro počáteční stav $A$:
\begin{itemize}[label={},noitemsep]
	\item $\begin{aligned} B &= cB + bC = c^*bC = c^*b(aB + bA + \varepsilon) = c^*baB + c^*bbA + c^*b = (c^*ba)^*(c^*bbA + c^*b)
	      \\ A &= aB + cB = (a + c)B = (a + c)(c^*ba)^*(c^*bbA + c^*b) =
	      (a + c)(c^*ba)^*c^*bbA + (a + c)(c^*ba)^*c^*b \\&= ((a + c)(c^*ba)^*c^*bb)^*
	      (a + c)(c^*ba)^*c^*b \end{aligned}$
\end{itemize}

\section{Příklad}
\subsection*{Zadání}
\begin{itemize}[label={},noitemsep]
	\item $ L_1 \subseteq \{a, b, c\}^*$
	\item $ L_2 \subseteq \{0, 1\}^* $
	\item $ bin: \{a, b, c\}^* \rightarrow \{0, 1\}^* $:
	      \begin{itemize}[label={},noitemsep]
	      	\item $ bin(\varepsilon) = \varepsilon $
	      	\item $ bin(a) = 00 $
	      	\item $ bin(b) = 01 $
	      	\item $ bin(c) = 11 $
	      \end{itemize}
	\item pro $ |w| > 1:  (w = xw' \wedge |x| = 1) \Rightarrow bin(w) = bin(x) bin(w') $
\end{itemize}

\begin{multicols}{2}
	\subsubsection*{Vstup}
	\begin{itemize}[label={},noitemsep]
		\item $ M_1 = (Q_1, \{a, b, c\}, \delta_1, q_1^0, F_1)$
		\item $ M_2 = (Q_2, \{0, 1\}, \delta_2, q_2^0, F_2)$
	\end{itemize}
	\columnbreak
	\subsubsection*{Výstup}
	\begin{itemize}[label={},noitemsep]
		\item $ M_f $, kde:
		\item $ L(M_f) = \{w | w \in l(M_1) \wedge bin(w) \in L(M_2)\} $
	\end{itemize}
\end{multicols}

\subsection*{Řešení}
$ M_f = (Q_f, \Sigma_f, \delta_f, q_f^0, F_f)$, kde:
\begin{enumerate}
	\item $ Q_f = Q_1 \times Q_2 $, automat používá stavy obou vstupních automatů
	\item $ \Sigma_f = \{a, b, c\}$, z definice výstupního automatu plyne, že přijímaná abeceda je stejná jako pro $ M_1 $
	\item $ \delta_f: Q_f \times \Sigma_f \rightarrow 2^{Q_f} $: \\
	      Automat pro přijetí znaku vyžaduje existenci přechodu v $ M_1 $ pomocí znaku vstupní abecedy a zároveň
	      je žádoucí, aby existoval přechod v $ M_2 $ pomocí binární reprezentace (výstup funkce $ bin $) toho
	      samého znaku. Ta však zahrnuje dva znaky abecedy přijímané $ M_2 $, a proto je třeba nalézt dva
	      navazující přechody v $ M_2 $. Formálně:
	      $\begin{aligned} \forall q_1^1, &q_1^2 \in Q_1, \forall q_2^1, q_2^2 \in Q_2, \forall a \in \Sigma_f: \\& (q_1^2, q_2^2) \in \delta_f((q_1^1, q_2^1), a) \Leftrightarrow \\& q_1^2 \in \delta_1(q_1^1, a) \wedge ( \forall x, y \in \{0, 1\}, \exists q_2^T \in Q_2: q_2^2 \in \delta_2(q_2^T, y) \wedge  q_2^T \in \delta_2(q_2^1, x)\ |\ xy = bin(a)) \end{aligned}$
	\item $ q_f^0 = (q_1^0, q_2^0) $, počáteční stav je společným počátečním stavem obou vstupních automatů
	\item $ F_f = F_1 \times F_2 $, koncové stavy jsou kartézským součinem koncových stavů obou vstupních automatů
\end{enumerate}

\section{Příklad}
\subsection*{Zadání}
$ L = \{a^ib^jc^k\ |\ i > k \wedge j \geq 1\}$, dokažte, je-li $L$ regulární.
\subsection*{Řešení}
\subsubsection*{Důkaz sporem}
Předpokládejme, že $L$ je regulární.
\bigskip

\noindent Pumping Lemma: $\exists p > 0, \forall w \in L: |w| \geq p \Rightarrow \exists x,y,z \in \Sigma^*: w = xyz \wedge 0 \leq |y| \leq k \wedge xy^lz \in L\ |\ l \geq 0$
\bigskip

\noindent Zvolme: $ w = a^{p+1}bc^p, w \in L: |w| = 2p+2 \geq p $
\bigskip

\noindent Pro $i = 2$ zvolme $w = xyz$:
\begin{itemize}[label={},noitemsep]
	\item $x = a^{p+1}b$
	\item $y = c^p$
	\item $z = \varepsilon$
\end{itemize}

\noindent Pak, ale zjistíme, že: $xy^lz = a^{p+1}bc^p$, kde platí:
\begin{itemize}[label={},noitemsep]
	\item $|x| = p + 2$, kde $x = a^{p+1}b$
	\item $|y| = 2p$, kde $y = c^{2p}$
	\item $|z| = 0$, kde $z = \varepsilon$
\end{itemize}

\noindent To znamená, že v řetězci $x$ je $p + 1$ symbolů \myuv{$a$} a v řetězci $y$ je $2p$ symbolů \myuv{$c$}. Tedy:
\begin{itemize}[label={},noitemsep]
	\item $i = p + 1$
	\item $k = 2p$
\end{itemize}
\noindent Musí platit: $ i > k \Rightarrow p + 1 > 2p \wedge p > 0$. Spor.

\section{Příklad}
\subsection*{Zadání}
Dokažte, že pro každý regulární jazyk existuje jednoznačná gramatika.
\subsection*{Řešení}
Každý regulární jazyk lze popsat konečným automatem. Víme, že pro každý konečný automat (\textit{KA}) existuje
ekvivalentní deterministický konečný automat (\textit{DKA}). To znamená, že v každé konfiguraci existuje jednoznačná
přechodová funkce a každý takový DKA je možné převést na pravou regulární gramatiku (\textit{G}). Z jednoznačnosti
přechodové funkce plyne, že výsledná gramatika generuje jednoznačný derivační strom, tedy z existence takové vazby mezi
\textit{KA}, \textit{DKA} až k \textit{G} je jasné, že existuje pro každá regulární jazyk jednoznačná gramatika.

\end{document}
